% e-book
\documentclass[oneside,12pt]{book}
\usepackage[paperwidth=210mm,paperheight=148mm,margin=5mm]{geometry}
\usepackage[unicode,colorlinks=true,pdfborderstyle={/S/U/W 1}]{hyperref}
\usepackage[pdftex]{graphicx}
\newcommand{\fig}[3]{\includegraphics[#3]{#2}}
% font setup for screen reading
\renewcommand{\familydefault}{\sfdefault}
\normalfont

% colors
\usepackage{xcolor}
\definecolor{darkcyan}{rgb}{0,.5,.5}

% Cyrillization
\usepackage[T1,T2A]{fontenc}
\usepackage[utf8]{inputenc}
%\usepackage[cp1251]{inputenc}
\usepackage[english,russian]{babel}
\usepackage{indentfirst}

% relative sectioning
\usepackage{ifthen}
\newcounter{secdepth}\setcounter{secdepth}{0}
\newcommand{\secup}{\addtocounter{secdepth}{1}}
\newcommand{\secdown}{\addtocounter{secdepth}{-1}}
\newcommand{\secrel}[1]{
\ifthenelse{\equal{\value{secdepth}}{0}}{\part{#1}}{}
\ifthenelse{\equal{\value{secdepth}}{-1}}{\chapter{#1}}{}
\ifthenelse{\equal{\value{secdepth}}{-2}}{\section{#1}}{}
\ifthenelse{\equal{\value{secdepth}}{-3}}{\subsection{#1}}{}
\ifthenelse{\equal{\value{secdepth}}{-4}}{\subsubsection{#1}}{}
}
\newcommand{\secly}[1]{\section*{#1}\addcontentsline{toc}{section}{#1}}

% misc
\bibliographystyle{acm}

\newcommand{\email}[1]{$<$\href{mailto:#1}{#1}$>$}
\usepackage{framed}		% boxed text
\usepackage{enumitem}	% [nosep] option in lists

\renewcommand{\emph}[1]{\textcolor{blue}{#1}}
\newcommand{\term}[1]{\textcolor{blue}{#1}}
\newcommand{\termdef}[2]{\term{#2}}
\newcommand{\note}[1]{\footnote{\ #1}}

% listing and computers
\usepackage{verbatim}
\usepackage{listings}
\lstset{
basicstyle=\small,
frame=single, % show frames around
numbers=left, numberstyle=\small,numbersep=1mm,% line numbering
tabsize=4, % tab style
%extendedchars=true,inputencoding=utf8, % i18n
keywordstyle=\color{magenta},%\texttt,
commentstyle=\color{darkcyan}
}
\newcommand{\pack}[1]{\textbf{\textcolor{blue}{#1}}} % program / package
\newcommand{\prog}[1]{\textbf{\textcolor{blue}{#1}}}
\newcommand{\file}[1]{\textbf{#1}}
\newcommand{\lst}[3]{\lstinputlisting[title=#1,language=#3]{#2}}
\newcommand{\lstx}[2]{\lstinputlisting[title=#1]{#2}}
%% languages
\newcommand{\cpp}{$C^{++}$}
\newcommand{\py}{Python}
\newcommand{\java}{Java}
\newcommand{\cs}{$C^\#$}
\newcommand{\bi}{Ы}
\newcommand{\lisp}{Lisp}
\newcommand{\st}{SmallTalk}
\newcommand{\forth}{Forth}