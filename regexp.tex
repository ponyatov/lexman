\secrel{Регулярные выражения}\label{regexp}

\termdef{Регулярное выражение}{Регулярное выражение} \cite{wikiregexp}, или 
\termdef{regexp}{regexp}\ --- текстовая строка, используемая в качестве
шаблона для проверки другой строки на совпадение, или поиска подстрок по
шаблону.

Большинство букв и символов соответствуют сами себе. Например, регулярное 
выражение \verb|test|\ будет в точности соответствовать строке \verb|test|.
Некоторые символы это специальные \emph{метасимволы}, и сами себе не 
соответствуют:

\begin{description}

\item{\verb|[  ]|} используются для определения набора символов, в виде
отдельных символов или диапазона, например regexp \verb|[0-9A-F]| задает одну
цифру шестнадцатеричного числа; набор \verb|[abcd]|\ можно заменить на диапазон
\verb|[a-d]|. 

\item{\verb|\|\ (обратная косая черта)} используется для \term{экранирования}\
специальных симолов, для представления как текстовых симолов самих по себе

\item{\verb|.|\ (точка)} обозначает любой символ, кроме конца строки

\item{\verb|^|} начало строки

\item{\verb|\n|} конец строки

\item{\verb|\t|} символ табуляции

\item{\verb|(  )|} скобки используются для определения области действия

\item{\verb$|$} вертикальная черта разделяет допустимые варианты,
часто используется вместе со скобками: \verb$gr(a|e)y$\ описывают строку 
\verb|gray|\ или \verb|grey|

\item{\termdef{квантификатор}{квантификатор}} после символа, 
символьного класса или группы 
определяет, сколько раз предшествующее выражение может встречаться

\item{\verb|{n}|} n раз
\item{\verb|{n,m}|} от n до m раз
\item{\verb|{n,}|} не менее n раз
\item{\verb|{,m}|} не более m раз

\item{\verb|?|} \verb|{0,1}| необязательный элемент
\item{\verb|*|} \verb|{0,}| 0+ раз
\item{\verb|+|} \verb|{1,}| 1+ раз

\end{description}

\subsubsection{Жадная и ленивая квантификация}

\begin{framed}
Квантификаторам в регулярных выражениях соответствует максимально длинная 
строка из возможных (квантификаторы являются \term{жадными} (greedy).
\end{framed} 

Это может оказаться значительной проблемой. Например, часто ожидают, что 
выражение \verb|(<.*>)|\ найдёт в тексте теги HTML. 
Однако, если в тексте есть более 
одного HTML-тега, то этому выражению соответствует целиком строка, содержащая 
множество тегов.

Эту проблему можно решить двумя способами:

\begin{enumerate}

\item Учитывать символы, не соответствующие желаемому образцу, через отрицание
в наборе символов: \verb|<[^>]*>|.

\item Определить квантификатор как \term{ленивый} (lazy)\ --- 
большинство реализаций обработчиков регулярных выражений позволяют это
сделать, добавив после квантификатора знак вопроса: \verb|<.*?>|

\end{enumerate}

Использование ленивых квантификаторов может повлечь за собой обратную проблему,
когда выражению соответствует слишком короткая, в частности, пустая строка.

