\secrel{Синтаксис и реализация языка \bi}\label{bisyntax}\secdown

\secrel{Комментарий}\label{bicomment}
\lstx{script.bI}{tmp/bi.comment}
Комментарии вырезаются лексером:
\lst{lpp.lpp}{tmp/lpp.comment}{c++}


\secrel{AST: абстрактный символьный тип}

\lst{hpp.hpp}{tmp/hpp.sym}{c++}
\lst{hpp.hpp}{tmp/hpp.ssm}{c++}

\secrel{Скалярные типы}\secdown

\secrel{Строка <str:'строка'>}

Работа лексера по разбору строк описана в \ref{lexstring}

\lst{lpp.lpp}{tmp/lpp.str}{c++}

\lst{ypp.ypp}{tmp/ypp.str}{c++}

\lst{hpp.hpp}{tmp/hpp.str}{c++}

\secrel{Числа}

Работа лексера по разбору чисел описана в \ref{lexnumbers}

\lst{ypp.ypp}{tmp/ypp.num}{c++}

\secrel{Int: целое число <int:1234>}

\lstx{script.bI}{tmp/bi.int}

\lst{lpp.lpp}{tmp/lpp.int}{c++}

\lst{hpp.hpp}{tmp/hpp.int}{c++}

\secrel{Hex: машинное шестнадцатеричное <hex:0x12AF> }

\lstx{script.bI}{tmp/bi.hex}

\lst{lpp.lpp}{tmp/lpp.hex}{c++}

\lst{hpp.hpp}{tmp/hpp.hex}{c++}

\secrel{Bin: машинное двоичное <bin:0b1101>}

\lstx{script.bI}{tmp/bi.bin}

\lst{lpp.lpp}{tmp/lpp.bin}{c++}

\lst{hpp.hpp}{tmp/hpp.bin}{c++}

\secrel{Num: число с плавающей точкой <num:1.23> <num:-3e+5>}

\lstx{script.bI}{tmp/bi.num}

\lst{lpp.lpp}{tmp/lpp.num}{c++}

\lst{hpp.hpp}{tmp/hpp.num}{c++}

\secup
\secup
