\secrel{Схема файлов для связки \prog{flex/bison}}

Сложные техники работы с текстовыми данными в этой книге будут далее 
рассматриваться на примере скриптового языка \bi. В этом разделе
рассмотрен вариант лексера, работающего в связке с генератором синтаксических
анализаторов \prog{bison}\ \ref{parser}. Такая связка\ --- типичная схема
построения синтаксического анализатора, способного разбирать многоуровневые
синтаксические конструкции.

\begin{framed}
Если вам требуется разбирать вложенные выражения, начиная от арифметических
выражений с инфиксными операторами и скобками, типа \verb|(1+2*3)/sin(x)|,
вам необходимо использовать связку \verb|flex+bison|\ (\verb|lex+yacc|)   
\end{framed}

\begin{tabular}{l}
в верхней части (до линии) перечислены определенные в файле объекты,\\
\hline
после линии указаны используемые объекты из других файлов. 
\end{tabular}

\subsubsection{ypp.ypp}

\begin{tabular}{l l l}
int & yyparse() & запуск парсера \\
union & yylval \verb|{| int i; float f; std::string *s; sym*o; \verb|}| &
структура для одного синтаксическогоо узла \\
\hline
void & yyerror(std::string) &
функция вызывается парсером\\&& при возникновении синтаксической ошибки \\
\end{tabular}

\subsubsection{lpp.lpp}

\begin{tabular}{l l l}
int & yylex() & функция лексера, выделяет \emph{один}\ токен в \verb|yylval|\\
&&и возвращает код токена, определенный в ypp.ypp.\\
char* & yytext & указатель на текст токена, выделенный лексером \\
int & yylength & длина выделенного текста\\
int & yylineno & номер текущей строки, требует \verb|%option yylineno|,
используется в yyerror()\\
\end{tabular}

\subsubsection{hpp.hpp}

\begin{tabular}{l l l l}
class & sym & \ref{sym} &
базовый виртуальный класс для символьных типов данных \\
&&&\underline{скалярые типы данных}:\\
sym & Sym & \ref{Sym} & символ \\
sym & Str & \ref{Str} & строка \\
sym & Int & \ref{Int} & целое число \\
sym & Num & \ref{Num} & число с плавающей точкой \\
&&&\underline{функциональные типы данных}:\\
sym & Op & \ref{Op} & оператор \\
&&&\\
\#define & TOC(C,X) & \ref{TOC} & макрос, создающий объект класса С для токена X \\
\end{tabular}

\subsubsection{cpp.cpp}

\begin{tabular}{l l l l}
int & main(int argc, char *argv\verb|[ ]|) &&\\
void & yyerror(std::string) & \ref{yyerror} &
функция аварийного завершения по ошибке \\
\hline
\verb|yyparse()|\\
\end{tabular}

\secrel{Лексер для языка \bi: script/lpp.lpp}\label{bilex}

Так как используется модульная компиляция, в файле \file{hpp.hpp}\ вынесены
объявления, которые нам нужно подключить: 

\lst{lpp.lpp}{tmp/lpp.hpp}{c++}

Макрос \verb|TOC|\ выполняет токенизацию символьного объекта \ref{symbol},
возвращая ссылку на объект и код токена в парсер: 

\lst{hpp.hpp}{tmp/hpp.TOC}{c++}\label{TOC}

\bigskip
Для компиляции и вывода номера строк в синтаксических ошибках
нужно включить пару опций:
 
\lstx{lpp.lpp}{tmp/lpp.options}

\subsubsection{Комментарий}

\lst{lpp.lpp}{tmp/lpp.comment}{c++}

\subsubsection{Строка}\label{lexstring}

Для разбора строк будет использоваться специальное состояние лексера,
и отдельный буфер разбора:  

\lst{lpp.lpp}{tmp/lpp.str}{c++}

В области определений задана строка\ --- буфер для накомпления символов строки
при ее разборе.

Через \verb|%x|\ создано \termdef{состояние лексера}{состояние лексера},
в области правил для этого состояния через \verb|<состояние>|\ задаются 
специальные \emph{правила, срабатывающие только для этого состояния}.

Состояния лексера переключаются макросом \verb|BEGIN()|, состояние по 
умолчанию\ --- \verb|INITIAL|.

\begin{description}
\item{\verb|'|} переключает состояние лексера, и обнуляет накопительный буфер
\item{\verb|<stringstate>'|} выключает состояние \verb|stringstate|,
и реализует работу макроса \verb|TOC(C,X)|\ особым образом, создавая объект
\verb|Str|\ из содержимого буфера, а не строки \verb|*yytext|
\item{\verb|<stringstate>.|} добавляет в буфер (.)=любой символ
\end{description}

\subsubsection{Числа}\label{lexnumbers}

\lst{lpp.lpp}{tmp/lpp.number}{c++}

При распозанавании чисел используется подставновка regexp-переменных
\verb|{S}|\ (знак числа) и \verb|{N}|\ (цифры), заданных в области определений. 

\subsubsection{Символ}\label{lexsymbol}

Все нераспознанные блоки текста, состоящие из литинских букв и цифр,
распознаются как \term{символ}\ \ref{symbol}:

\lst{lpp.lpp}{tmp/lpp.sym}{c++}

\subsubsection{Операторы}\label{lexops}

\lst{lpp.lpp}{tmp/lpp.ops}{c++}


\secup