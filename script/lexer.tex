\secrel{Лексер для языка \bi}\secdown

Сложные техники работы с текстовыми данными в этой книге будут далее 
рассматриваться на примере скриптового языка \bi. В этом разделе
рассмотрен вариант лексера, работающего в связке с генератором синтаксических
анализаторов \prog{bison}\ \ref{parser}. Такая связка\ --- типичная схема
построения синтаксического анализатора, способного разбирать многоуровневые
синтаксические конструкции.

\begin{framed}
Если вам требуется разбирать вложенные выражения, начиная от арифметических
выражений с инфиксными операторами и скобками, типа \verb|(1+2*3)/sin(x)|,
вам необходимо использовать связку \verb|flex+bison|\ (\verb|lex+yacc|)   
\end{framed}

\secrel{Схема файлов для связки \prog{flex/bison}}\secdown

%\fig{}{fig/multi.png}{width=0.9\textwidth}

\begin{tabular}{l}
в верхней части (до линии) перечислены определенные в файле объекты,\\
\hline
после линии указаны используемые объекты из других файлов. 
\end{tabular}

\secrel{ypp.ypp}

\begin{tabular}{l l l}
int & yyparse() & запуск парсера \\
union & yylval \verb|{| int i; float f; std::string *s; sym*o; \verb|}| &
структура для одного синтаксическогоо узла \\
\hline
void & yyerror(std::string) &
функция вызывается парсером\\&& при возникновении синтаксической ошибки \\
\end{tabular}

\secrel{lpp.lpp}

\begin{tabular}{l l l}
int & yylex() & функция лексера, выделяет \emph{один}\ токен в \verb|yylval|\\
&&и возвращает код токена, определенный в ypp.ypp.\\
char* & yytext & указатель на текст токена, выделенный лексером \\
int & yylength & длина выделенного текста\\
int & yylineno & номер текущей строки, требует \verb|%option yylineno|,
используется в yyerror()\\
\end{tabular}

\secrel{hpp.hpp}

\begin{tabular}{l l l l}
class & sym & \ref{sym} &
базовый виртуальный класс для символьных типов данных \\
&&&\underline{скалярые типы данных}:\\
sym & Sym & \ref{Sym} & символ \\
sym & Str & \ref{Str} & строка \\
sym & Int & \ref{Int} & целое число \\
sym & Num & \ref{Num} & число с плавающей точкой \\
&&&\underline{функциональные типы данных}:\\
sym & Op & \ref{Op} & оператор \\
\#define & TOC(C,X) & \ref{TOC} & макрос, создающий объект класса С для токена X \\
\end{tabular}

\secrel{cpp.cpp}

\begin{tabular}{l l l l}
int & main(int argc, char *argv\verb|[ ]|) &&\\
void & yyerror(std::string) & \ref{yyerror} &
функция аварийного завершения по ошибке \\
\hline
\verb|yyparse()|\\
\end{tabular}


\secup


\secrel{\file{lpp.lpp}} 

\lst{lpp.lpp}{script/lpp.lpp}{c++}

\secup