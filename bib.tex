
\begin{thebibliography}{99}
\addcontentsline{toc}{chapter}{Литература}

\secly{Основы компиляторов}

\bibitem{dragon}
\href{https://drive.google.com/file/d/0B0u4WeMjO894LS1Jb3JUbEVSVkE/view?usp=sharing}{Книга Дракона (Dragon Book)}:\\
Альфред Ахо, Моника С. Лам, Рави Сети, Джеффри Ульман\\
\textbf{Компиляторы: принципы, технологии и инструментарий}


\bibitem{habr1} Habrahabr: \href{http://habrahabr.ru/post/99162/}{Компиляция. 1: лексер}

\bibitem{habr2} Habrahabr: \href{http://habrahabr.ru/post/99298/}{Компиляция. 2: грамматики}

\secly{LLVM}

\bibitem{llvm} \href{http://llvm.org/docs/tutorial/}{LLVM tutorial}

\secly{Java/ANTLR}

\bibitem{anthabr} \href{http://habrahabr.ru/post/110710/}{Грамматика 
арифметики или пишем калькулятор на ANTLR}

\secly{Утилиты}

\bibitem{graphviz}
Emden Gansner and Eleftherios Koutsofios and Stephen North\\
\textbf{\href{http://www.graphviz.org/Documentation/dotguide.pdf}{Drawing graphs with dot}}

\secly{\LaTeX: система верстки для научных публикаций}

\bibitem{wikilatex} Википедия:
\href{https://ru.wikipedia.org/wiki/LaTeX}{система верстки \LaTeX}

\bibitem{tex}
Котельников И. А., Чеботаев П. З.\\ 
\textbf{\LaTeX по-русски}.\\
— СПб.: «Корона-Век», 2011. — 496 с. — 2000 экз. — ISBN 978-5-7931-0878-2.

\bibitem{texlvovsky}
Львовский С. М.\\
\href{http://www.mccme.ru/free-books/llang/newllang.pdf}{\textbf{Набор и верстка в системе LaTeX.}}\\
— М.: МЦНМО, 2006. — С. 448. — ISBN 5-94057-091-7.

\bibitem{texbaldin}
\href{http://www.inp.nsk.su/~baldin/LaTeX/}{Балдин Е. М.}\\
\href{http://mirrors.ctan.org/info/russian/Computer_Typesetting_Using_LaTeX/ctex.pdf}{\textbf{Компьютерная типография LaTeX.}}\\
 — «БХВ-Петербург», 2008. — 304 с. — 2000 экз. — ISBN 978-5-9775-0230-6. Книга доступна в электроном виде на сайте CTAN под лицензией CC-BY-SA.

\bibitem{texdiplom}
Столяров А. В.\\
\textbf{Сверстай диплом красиво: LaTeX за три дня.}\\
 — Москва: МАКС Пресс, 2010. — 100 с. — 200 экз. — ISBN 978-5-317-03440-5.

\end{thebibliography}