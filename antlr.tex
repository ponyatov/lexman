\secrel{\java, \andr, \cs, ANTLR}\secdown

В современных IT-приложениях не всегда возможно использовать язык
\cpp\ и динамическую компиляцию. Для случаев, когда вы вынуждены использовать
среды, заточенные под \java\ или .NET/\cs, можно использовать генератор 
парсеров ANTLR. В этом разделе будет рассмотрен простая языковая 
среда-калькулятор, рализованная в нескольких вариантах, в т.ч. как приложение
\andr.

\cp{\cite{anthabr}}

Когда встает вопрос о том, как в программе на \java\ или \cs\ разобрать и 
вычислить арифметическое выражение, заданное в
виде привычного инфиксного выражения, можно составить грамматику 
входного языка, и построить по ней лексический и синтаксический 
анализатор при помощи \href{http://www.antlr.org/}{ANTLR}.

\begin{framed}
\term{[AN]other [T]ool for [L]anguage [R]ecognition)}\ --- мощный генератор
парсеров для чтения, обработки, выполнения и трансляции
структурированных текстовых или бинарных файлов. \term{ANTLR}\ широко 
используется для построения языковых утилит или фреймворков. На основе 
грамматики ANTLR генерирует парсер, который может строить и обходить 
деревья разбора.
\end{framed}

ANTLR\ --- генератор парсеров, позволяющий автоматически создавать
программу-парсер (как и лексический анализатор) на одном из целевых языков
программирования (\cpp, \java, \cs, \py, Ruby) по описанию LL(*)-грамматики 
на языке, близком к РБНФ.

\secrel{Инсталяция}\secdown

\secrel{Java JDK}

Для самостоятельного использования платформы \java SE достаточно
32-битной версии, для разработки под \andr\ современная среда разработки Android 
Studio \ref{adkinstall}\ требует исключительно 64-битную версию JDK.

\bigskip
\url{http://www.oracle.com/technetwork/java/javase/downloads/index.html}
\bigskip

Скачайте и установите \java\ JDK x32/x64:

\menu{Java Platform, Standard Edition>JDK>Download}

\menu{Java SE Development Kit 8u65>Accept License agreement} 

\menu{\win\ x86>jdk-8u65-windows-i586.exe}

\menu{\win\ x64>\textbf{jdk-8u65-windows-x64.exe}}

\bigskip
После установки JDK нужно прописать несколько переменных окружения для 
пользователя:

\menu{\winstart>Компьютер>\rms>Свойства>Доп.параметры системы>Переменные среды}
\begin{verbatim}
JAVA_HOME = C:\Java\jdk64
PATH = %JAVA_HOME%\bin;%PATH%
\end{verbatim}

Переменная \verb|JAVA_HOME|\ необходима для правильной установки
Android Studio \ref{adkinstall}.

\secrel{ANTLR}

\subsubsection{\win}

\begin{verbatim}
cd \Java ; mkdir ANTLR ; cd ANTLR
wget -c http://www.antlr.org/download/antlr-4.5.1-complete.jar
wget -c 
\end{verbatim}

Установите \href{http://tunnelvisionlabs.com/products/demo/antlrworks}{ANTLRWorks},
распаковав \file{.zip}\ в \verb|C:\Java\ANTLR\|.
\bigskip

Если вы работаете в \eclipse, 
\href{https://github.com/jknack/antlr4ide}{плагин Edgar Espina}
доступен для установки через \eclipse\ Marketplace под именем
ANTLR 4 IDE в комплекте с самой библиотекой ANTLR, устанавливаемой как плагин.
\bigskip

Установите переменные окружения для пользователя \verb|CLASSPATH|\note{не 
забудьте про точку .;}\ и добавьте в \verb|PATH|:

\begin{verbatim}
CLASSPATH = .;C:\Java\ANTLR\antlr-4.5.1-complete.jar;%CLASSPATH%
PATH = C:\Java\ANTLR;%PATH%
\end{verbatim}

Cоздайте в \verb|\Java\jdk\bin\|\ пару батников для запуска утилит:

\begin{lstlisting}[title=\file{antlr4.bat}]
java org.antlr.v4.Tool %*
\end{lstlisting}
\begin{lstlisting}[title=\file{grun.bat}]
java org.antlr.v4.gui.TestRig %*
\end{lstlisting}

\secrel{Android Studio}\label{adkinstall}

\url{http://developer.android.com/intl/ru/sdk/index.html}\bigskip

\menu{All Android Studio Packages>Windows>android-studio-bundle-xxx.xxx-windows.exe}

\bigskip
Bundle включает не только IDE на базе IntelliJava, но и Android SDK.

\secup

\secrel{Необходимый инструментарий}

\begin{itemize}[nosep]
\item \prog{ANTLRWorks} --- среда для разработки и отладки грамматик
\item \prog{ANTLR}\ --- собственно сам ANTLR
\item \prog{Antlr4.Runtime.dll} \cs\ runtime distribution\ -- .NET библиотека 
для работы со сгенирированными анализаторами, входит в пакет 
\file{antlr-csharp-runtime-4.x.x.zip}
\end{itemize}

\secrel{Разработка грамматики}

Настало время запустить ANTLRWorks и создать проект с грамматикой калькулятора:

\menu{File>New File>ANTLR>ANTLR 4 Combined Grammar>Next}

\menu{Folder>\file{D:/w/lexman/Jcalc}}

\menu{File Name>ANTLRgrammar}

\menu{Finish}

\lstx{ANTLRgrammar.g4}{tmp/ant.gr}

Тип грамматики Combined Grammar означает, что в одном файле у нас будут 
находится одновременно и правила лексера, и синтаксического анализатора.

Давайте разберем, как записываются правила для лексера на примере правила:
целого числа \verb|INT|. 

Правило начинается с имени, \emph{для правил лексера}\ обязаны начинаться 
\emph{с заглавной буквы}. После имени идет символ \verb|:|, за которым следуют
\term{альтернативы}. Альтернативы разделяются сиволом \verb$|$\ и должны 
заканчиваться символом \verb|;|.

Посмотрим на единственную альтернативу для правила \verb|INT|. 
Запись \verb|'0'..'9'+| в ANTLR означает, что \verb|INT|\ --- непустая
последовательность цифр. Символ \verb|*|\ допускает пустую последовательность.

\lstx{ANTLRgrammar.g4}{tmp/ant.int}

Правило \verb|ID|\ несколько сложнее и означает, что \verb|ID|\ --— это 
последовательность символов, состоящая из строчных и заглавных латинских букв, 
цифр, символов подчеркивания и начинающаяся с буквы или символа подчеркивания.

\lstx{ANTLRgrammar.g4}{tmp/ant.id}

Для работы смешанной грамматики необходимо минимум одно правило-нетерминал:
первое правило для парсера или по-научному \term{аксиому}, то есть правило, 
с которого парсер начнет проверять поток лексем:

\lstx{ANTLRgrammar.g4}{tmp/ant.calc}

В данном случае мы говорим парсеру, что входной поток лексем представляет из 
себя непустую (\verb|+|) последовательность \verb|expr|ов, в одном из вариантов
включающий \term{скалярные типы}\ \verb|INT|\ и \verb|ID|.

\secup
