\secrel{Лексер и утилита \file{flex}}\label{lexer}\secdown

\termdef{лексер}{Лексер}\ выполняет разбор входного потока единичных символов,
выделяя из него группы символов. Код лексера генерируется с помощью утилиты
\prog{flex}, из набора правил, состоящих из двух частей:

\begin{enumerate}
\item регулярное выражение \ref{regexp}, задающее шаблон для выделения группы 
символов, и
\item блок произвольного кода на \cpp, выполняющего с найденным текстом нужные
действия.
\end{enumerate}

Для простых применений вы можете прописать нужные вам действия непосредственно
с заданным текстом (запись в отдельный файл, преобразования,\ldots).

\begin{framed}
В случае использования лексера в составе транслятора/комипилятора, лексер
выполняет \termdef{токенизация}{токенизацию}: первичное преобразование 
найденных блоков исходного текста в \term{токены}.
\end{framed} 

\secrel{Структура файла описания лексера}

Для утилиты \prog{flex}\note{или ее предшественника \prog{lex}} используется
файл с расширением \verb|.l|/\verb|.lex|/\verb|.lpp|\ \cite{openlex}:

\begin{verbatim}
// секция определений
    %{
        // заголовочный C++ код
        #include "hpp.hpp"
        #include "parser.tab.hpp"
        std::string StringParseBuffer;
    %}
    // опции
        %option ...
    // дополнительные состояния лексера
        %x state1
        %s state2
// секция правил
    %%
    %%
// секция подпрограмм
\end{verbatim}

Минимальный вариант .lex-файла:

\begin{verbatim}
%option main    // добавить автоматическую функцию main() 
%%
...    // правила
\n     {<код для конца строки>} или {}
.      {<код для нераспознанного символа>}, {}
%%
\end{verbatim}

\secrel{Запуск \prog{flex}}\secdown

\secrel{Запуск в варианте для старого \prog{lex}}

Для начала рассмотрим вариант использования для старой версии
лексического генератора \prog{lex}, который вы внезапно встретите
в какой-нибудь старой коммерческой UNIX-системе. Подробно отличия 
рассмотрены в \cite{lextoflex}.

\lstx{empty.l}{tmp/empty.1}

\begin{verbatim}
lex empty.l
cc -o empty.exe lex.yy.c
./empty.exe < lex.yy.c > empty.log 
\end{verbatim}

На новых UNIXах аналогичного результата можно добиться командами, 
включающими режим совместимости со старыми версиями ПО:

\begin{verbatim}
flex -l empty.l
gcc -std=c89 -Wpedantic -o empty.exe lex.yy.c
./empty.exe < lex.yy.c > empty.log 
\end{verbatim}

После выполнения команды \verb|lex|\ будет создан 
файл \verb|lex.yy.c|, содержащий \emph{чисто сишный}\ код лексера, 
\emph{который можно откомпилировать любым ANSI-совместимым компилятором Си
для любого микроконтроллера}, или отечественной ВПКшной поделки типа 
\href{http://www.angstrem.ru/products/micro/tesey-8/KP1878BE1.html}{КР1878ВЕ1}.

\begin{framed}
Полученная программа читает символы с \verb|stdin|, и выводит все 
нераспознанные символы на \verb|stdout|.
\end{framed}

Сравнив файлы \file{lex.yy.c}\ и \file{empty.log}, вы увидите что они 
полностью совпадают. Чтобы сделать что-то типа полезное, добавим несколько
правил, и получим список команд препроцессора, характерных для языка Си:

В конец набора правил добавим удаление пробельных символов 
и нераспознанных символов:

\lstx{empty.l}{tmp/empty.sp}

\begin{lstlisting}[title=empty.log]
\end{lstlisting}

В итоге мы получили пустой файл, так как были удалены все символы.
Теперь пользуясь справочником по языку Си, 
добавим \emph{в начало списка}\ правило,
используя \term{регулярное выражение}\note{подробно рассмотрены 
далее \ref{regexp}}\ для команд препроцессора:

\lstx{empty.l}{tmp/empty.2}
\lstx{empty.log}{tmp/empty.log}

В результате на выходе мы получили все части строк от символа \verb|#|\ до
конца строки \verb|\n|, между которыми находится 1+ любых символов \verb|.+|.

\secrel{\prog{flex}\ и генерация лексера на \cpp}

Если вы пишите лексический анализатор для компьютера, а не микроконтроллера,
это удобнее делать на \cpp. 
\emph{Если вы пишете на \cpp, лучше использовать расширения файлов .lpp}.
Это расширение также укажет на то, что полученный генератор ограниченно
применим для микроконтроллера: код на \cpp\ для МК требует очень аккуратной
работы с динамической памятью из-за малого объема ОЗУ. 

Современный генератор анализаторов\
\prog{flex}\ поддерживает два варианта генерации кода, совместимого с \cpp:

\begin{enumerate}
\item использовать традиционный запуск \verb|flex empty.lpp|, но компилировать
полученный \file{lex.yy.c}\ компилятором \prog{g++}: в этом случае вы можете
свободно использовать в правилах код на \cpp, но весь ввод/вывод будет работать
через файлы Си \verb|FILE* stdin,stdout|, а не через потоки. 

\lstx{empty.lpp}{empty.lpp}

\item запускать \verb|flex -+ empty.lpp|, \verb|flex++ empty.lpp|
или с \verb|%option c++|\ в .lpp файле: анализатор будет сгенерирован в файл
\file{lex.yy.cc}, и требует от вас создания файла \file{FlexLexer.h},
содержащего определения пары служебных классов для лексера. Детали
использования \verb|flex++|\ рассмотрены в \ref{flexpp}. 

\end{enumerate}  

\secup

\secrel{Регулярные выражения}\label{regexp}

\termdef{Регулярное выражение}{Регулярное выражение}, или 
\termdef{regexp}{regexp}\ --- текстовая строка, используемая в качестве
шаблона для проверки другой строки на совпадение, или поиска подстрок по
шаблону.

Большинство букв и символов соответствуют сами себе. Например, регулярное 
выражение \verb|test|\ будет в точности соответствовать строке \verb|test|.
Некоторые символы это специальные \emph{метасимволы}, и сами себе не 
соответствуют:

\begin{description}
\item{\verb|[  ]|} используются для определения набора символов, в виде
отдельных символов или диапазона, например regexp \verb|[0-9A-F]| задает одну
цифру шестнадцатеричного числа; набор \verb|[abcd]|\ можно заменить на диапазон
\verb|[a-d]|. 
\end{description}

\secrel{Примеры самостоятельного применения}\secdown

\begin{framed}
\noindent Лексер может быть использован как самостоятельный инструмент, если
не требуется анализ синтаксиса, и достаточно выполнять заданный \cpp\ код
при срабатывании одного из регулярных выражений.
\end{framed}

\secrel{\file{Pij2D}: загрузка файла числовых данных}

Формат файла: \bigskip

\begin{itemize}[nosep]
\item число строк матрицы max=Rmax
\item число элементов в строке max=Xmax
\item данные построчно
\end{itemize}

\lstx{Fi.dat}{tmp/Fi.dat}
\lstx{Pij2D.lpp}{../pij/pij2d/lpp.lpp}
\lst{hpp.hpp}{../pij/pij2d/hpp.hpp}{c++}
\lst{cpp.cpp}{tmp/pij.main}{c++}

\begin{verbatim}
Rmax 					// строк, не более чем 
Xmax 					// столбцов, не более чем
double Fi[Rmax][Xmax]	// массив под данные
int item				// общий счетчик прочитанных чисел

argc, argv				// часть исходных данных задается с командной строки
doit()					// функция обработки данных

while (yylex());		// цикл опроса лексера,
yylex()					// на каждый вызов возвращается один токен
\end{verbatim}

\verb|item|\ используется для определения, какой тип имеет текущее 
прочитанное число: \verb|Rlimit|, \verb|Xlimit| или данные.

Конец строки в обработке не участвует, факт перехода на следующую строку

\subsubsection{Компиляция}

\verb|cd pij/pij2d && make|

\lst{Makefile}{../pij/pij2d/Makefile}{make}

verb|cd pij/pij2d|

\secup

\secrel{Схема файлов для связки \prog{flex/bison}}

Сложные техники работы с текстовыми данными в этой книге будут далее 
рассматриваться на примере скриптового языка \bi. В этом разделе
рассмотрен вариант лексера, работающего в связке с генератором синтаксических
анализаторов \prog{bison}\ \ref{parser}. Такая связка\ --- типичная схема
построения синтаксического анализатора, способного разбирать многоуровневые
синтаксические конструкции.

\begin{framed}
Если вам требуется разбирать вложенные выражения, начиная от арифметических
выражений с инфиксными операторами и скобками, типа \verb|(1+2*3)/sin(x)|,
вам необходимо использовать связку \verb|flex+bison|\ (\verb|lex+yacc|)   
\end{framed}

\begin{tabular}{l}
в верхней части (до линии) перечислены определенные в файле объекты,\\
\hline
после линии указаны используемые объекты из других файлов. 
\end{tabular}

\subsubsection{ypp.ypp}

\begin{tabular}{l l l}
int & yyparse() & запуск парсера \\
union & yylval \verb|{| int i; float f; std::string *s; sym*o; \verb|}| &
структура для одного синтаксическогоо узла \\
\hline
void & yyerror(std::string) &
функция вызывается парсером\\&& при возникновении синтаксической ошибки \\
\end{tabular}

\subsubsection{lpp.lpp}

\begin{tabular}{l l l}
int & yylex() & функция лексера, выделяет \emph{один}\ токен в \verb|yylval|\\
&&и возвращает код токена, определенный в ypp.ypp.\\
char* & yytext & указатель на текст токена, выделенный лексером \\
int & yylength & длина выделенного текста\\
int & yylineno & номер текущей строки, требует \verb|%option yylineno|,
используется в yyerror()\\
\end{tabular}

\subsubsection{hpp.hpp}

\begin{tabular}{l l l l}
class & sym & \ref{sym} &
базовый виртуальный класс для символьных типов данных \\
&&&\underline{скалярые типы данных}:\\
sym & Sym & \ref{Sym} & символ \\
sym & Str & \ref{Str} & строка \\
sym & Int & \ref{Int} & целое число \\
sym & Num & \ref{Num} & число с плавающей точкой \\
&&&\underline{функциональные типы данных}:\\
sym & Op & \ref{Op} & оператор \\
&&&\\
\#define & TOC(C,X) & \ref{TOC} & макрос, создающий объект класса С для токена X \\
\end{tabular}

\subsubsection{cpp.cpp}

\begin{tabular}{l l l l}
int & main(int argc, char *argv\verb|[ ]|) &&\\
void & yyerror(std::string) & \ref{yyerror} &
функция аварийного завершения по ошибке \\
\hline
\verb|yyparse()|\\
\end{tabular}

\secrel{Лексер для языка \bi: script/lpp.lpp}\label{bilex}

Так как используется модульная компиляция, в файле \file{hpp.hpp}\ вынесены
объявления, которые нам нужно подключить: 

\lst{lpp.lpp}{tmp/lpp.hpp}{c++}

Макрос \verb|TOC|\ выполняет токенизацию символьного объекта \ref{symbol},
возвращая ссылку на объект и код токена в парсер: 

\lst{hpp.hpp}{tmp/hpp.TOC}{c++}\label{TOC}

\bigskip
Для компиляции и вывода номера строк в синтаксических ошибках
нужно включить пару опций:
 
\lstx{lpp.lpp}{tmp/lpp.options}

\subsubsection{Комментарий}

\lst{lpp.lpp}{tmp/lpp.comment}{c++}

\subsubsection{Строка}\label{lexstring}

Для разбора строк будет использоваться специальное состояние лексера,
и отдельный буфер разбора:  

\lst{lpp.lpp}{tmp/lpp.str}{c++}

В области определений задана строка\ --- буфер для накомпления символов строки
при ее разборе.

Через \verb|%x|\ создано \termdef{состояние лексера}{состояние лексера},
в области правил для этого состояния через \verb|<состояние>|\ задаются 
специальные \emph{правила, срабатывающие только для этого состояния}.

Состояния лексера переключаются макросом \verb|BEGIN()|, состояние по 
умолчанию\ --- \verb|INITIAL|.

\begin{description}
\item{\verb|'|} переключает состояние лексера, и обнуляет накопительный буфер
\item{\verb|<stringstate>'|} выключает состояние \verb|stringstate|,
и реализует работу макроса \verb|TOC(C,X)|\ особым образом, создавая объект
\verb|Str|\ из содержимого буфера, а не строки \verb|*yytext|
\item{\verb|<stringstate>.|} добавляет в буфер (.)=любой символ
\end{description}

\subsubsection{Числа}\label{lexnumbers}

\lst{lpp.lpp}{tmp/lpp.number}{c++}

При распозанавании чисел используется подставновка regexp-переменных
\verb|{S}|\ (знак числа) и \verb|{N}|\ (цифры), заданных в области определений. 

\subsubsection{Символ}\label{lexsymbol}

Все нераспознанные блоки текста, состоящие из литинских букв и цифр,
распознаются как \term{символ}\ \ref{symbol}:

\lst{lpp.lpp}{tmp/lpp.sym}{c++}

\subsubsection{Операторы}\label{lexops}

\lst{lpp.lpp}{tmp/lpp.ops}{c++}


\secup
