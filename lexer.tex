\secrel{Лексер и утилита \file{flex}}\label{lexer}\secdown

\secrel{Регулярные выражения}\label{regexp}

\termdef{Регулярное выражение}{Регулярное выражение}, или 
\termdef{regexp}{regexp}\ --- текстовая строка, используемая в качестве
шаблона для проверки другой строки на совпадение, или поиска подстрок по
шаблону.

Большинство букв и символов соответствуют сами себе. Например, регулярное 
выражение \verb|test|\ будет в точности соответствовать строке \verb|test|.
Некоторые символы это специальные \emph{метасимволы}, и сами себе не 
соответствуют:

\begin{description}
\item{\verb|[  ]|} используются для определения набора символов, в виде
отдельных символов или диапазона, например regexp \verb|[0-9A-F]| задает одну
цифру шестнадцатеричного числа; набор \verb|[abcd]|\ можно заменить на диапазон
\verb|[a-d]|. 
\end{description}

\secrel{Примеры самостоятельного применения}\secdown

\begin{framed}
\noindent Лексер может быть использован как самостоятельный инструмент, если
не требуется анализ синтаксиса, и достаточно выполнять заданный \cpp\ код
при срабатывании одного из регулярных выражений.
\end{framed}

\secrel{\file{Pij2D}: загрузка файла числовых данных}

Формат файла: \bigskip

\begin{itemize}[nosep]
\item число строк матрицы max=Rmax
\item число элементов в строке max=Xmax
\item данные построчно
\end{itemize}

\lstx{Fi.dat}{tmp/Fi.txt}
\lstx{Pij2D.lpp}{../pij/pij2d/lpp.lpp}
\lst{hpp.hpp}{../pij/pij2d/hpp.hpp}{c++}

\begin{lstlisting}[language=c++]
int main (int argc, char *argv[]) {
        // command line processing
        assert (argc==4+1);     // pij.exe [V] [Qm] [Alpha]
        V     = atof(argv[1]); assert(V    >0); cout << "V:\t\t\t" << V << "\n";
        Qm    = atof(argv[2]); assert(Qm   >0); cout << "Qm:\t\t\t" << Qm << "\n";
        Alpha = atof(argv[3]); assert(Alpha>0); cout << "Alpha:\t\t" << Alpha << "\n";
        r         = atof(argv[4])/1000; assert(r    >0); cout << "r:\t\t\t" << r << "\n";
        // Fi.txt fielf data parsing
        while (yylex());        // Fi.txt parser loop from stdin
        // compute tracks
        return doit();
}
\end{lstlisting}

\begin{verbatim}
Rmax 					// строк, не более чем 
Xmax 					// столбцов, не более чем
double Fi[Rmax][Xmax]	// массив под данные
int item				// общий счетчик прочитанных чисел

argc, argv				// часть исходных данных задается с командной строки
doit()					// функция обработки данных

while (yylex());		// цикл опроса лексера,
yylex()					// на каждый вызов возвращается один токен
\end{verbatim}

\verb|item|\ используется для определения, какой тип имеет текущее 
прочитанное число: \verb|Rlimit|, \verb|Xlimit| или данные.

Конец строки в обработке не участвует, факт перехода на следующую строку
матрицы определяется по превышению Xlimit.

\secup
\secup