\secrel{Типичная структура проекта}\secdown

\begin{tabular}{l l l l}
\file{README.md} &\ref{readme}& github 
	& описание проекта на \url{https://github.com/} \\
\file{Makefile} &\ref{makefile}& make 
	& зависимости между файлами и команды сборки \\
\file{lpp.lpp} & \ref{lpp} & flex 
	& лексер \ref{lexer}\\
\file{ypp.ypp} & \ref{ypp} & bison 
	& парсер \ref{parser}\\
\file{hpp.hpp} & \ref{hpp} & g++/clang++ 
	& заголовочные файлы \cpp \\
\file{cpp.cpp} & \ref{cpp} & g++/clang++ 
	& \cpp-код: ядро интерпретатора, компилятор,\\
	&&& реализация динамических типов, пользовательский код \\
\file{bat.bat} & \ref{bat} & win32 
	& запускалка gvim\\
\file{rc.rc} & \ref{rc} & windres 
	& описание ресурсов: иконки приложения, меню,..\\
\file{logo.ico} && windres 
	& логотип в .ico формате \\
\file{logo.png} &&
	& логотип в .png (для github README) \\
\file{filetype.vim} & \ref{filetypevim} & (g)vim 
	& привязка расширения файлов cкриптов \\
\file{syntax.vim} & \ref{syntaxvim} & (g)vim 
	& синтаксическая подсветка для скриптов \\
\file{.gitignore} & \ref{gitignore} & git 
	& список временных и производных файлов \\
\end{tabular}

\secrel{README.md}\label{readme}

\begin{verbatim}
# <логотип> <название>
(c) <имя> <email>
<лицензия>
<ссылка на проект на GitHub>
### <ссылки, дополнительная информация>
\end{verbatim}
\begin{lstlisting}[title=README.md]
# ![logo](logo.png) Mega script language

(c) Vasya Pupkin <pupkin@gmail.com>, all rights reserved

license: http://www.gnu.org/copyleft/lesser.html

GitHub: https://github.com/pupkin/megascript
\end{lstlisting}
\secrel{Makefile}\label{makefile}

Опции сборки (win32|linux):

\bigskip
\begin{tabular}{l l}
EXE & суффикс исполняемого файла \\
RES & имя объектного файла ресурсов win32.exe \\
TAIL & опция команды \file{tail}\ число последних строк \file{MODULE.log} \\ 
\end{tabular}

\lst{Makefile}{tmp/mk.head}{make}

Модуль заполняется автоматически по имени текущего каталога:

\lst{Makefile}{tmp/mk.module}{make}

Цель команды \file{make}\ по умолчанию: сборка и интерпретация тестового файла

\lst{Makefile}{tmp/mk.exec}{make}



\secrel{bat.bat}\label{bat}

\lstx{bat.bat}{script/bat.bat}

\secrel{rc.rc}

\lstx{rc.rc}{script/rc.rc}

\secup