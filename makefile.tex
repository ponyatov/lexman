\secrel{Makefile}\label{makefile}

Опции сборки (win32|linux):

\bigskip
\begin{tabular}{l l}
EXE & суффикс исполняемого файла \\
RES & имя объектного файла ресурсов win32.exe \\
TAIL & опция команды \file{tail}\ число последних строк \file{MODULE.log} \\ 
\end{tabular}

\lst{Makefile}{tmp/mk.head}{make}

Модуль заполняется автоматически по имени текущего каталога:

\lst{Makefile}{tmp/mk.module}{make}

\emph{Цель команды \file{make}\ по умолчанию: сборка и интерпретация тестового файла}

\lst{Makefile}{tmp/mk.exec}{make}

Вторая (стандартная) цель \file{clean}: удаление временных и рабочих файлов

\lst{Makefile}{tmp/mk.clean}{make}

Сборка \cpp\ части

\lst{Makefile}{tmp/mk.cpp}{make}

Генерация кода парсера

\lst{Makefile}{tmp/mk.ypp}{make}

Генерация кода лексера

\lst{Makefile}{tmp/mk.lpp}{make}

Компиляция файла ресурсов (win32)
\lst{Makefile}{tmp/mk.res}{make}
